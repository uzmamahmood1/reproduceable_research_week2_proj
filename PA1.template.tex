\documentclass[]{article}
\usepackage{lmodern}
\usepackage{amssymb,amsmath}
\usepackage{ifxetex,ifluatex}
\usepackage{fixltx2e} % provides \textsubscript
\ifnum 0\ifxetex 1\fi\ifluatex 1\fi=0 % if pdftex
  \usepackage[T1]{fontenc}
  \usepackage[utf8]{inputenc}
\else % if luatex or xelatex
  \ifxetex
    \usepackage{mathspec}
  \else
    \usepackage{fontspec}
  \fi
  \defaultfontfeatures{Ligatures=TeX,Scale=MatchLowercase}
\fi
% use upquote if available, for straight quotes in verbatim environments
\IfFileExists{upquote.sty}{\usepackage{upquote}}{}
% use microtype if available
\IfFileExists{microtype.sty}{%
\usepackage{microtype}
\UseMicrotypeSet[protrusion]{basicmath} % disable protrusion for tt fonts
}{}
\usepackage[margin=1in]{geometry}
\usepackage{hyperref}
\hypersetup{unicode=true,
            pdfborder={0 0 0},
            breaklinks=true}
\urlstyle{same}  % don't use monospace font for urls
\usepackage{color}
\usepackage{fancyvrb}
\newcommand{\VerbBar}{|}
\newcommand{\VERB}{\Verb[commandchars=\\\{\}]}
\DefineVerbatimEnvironment{Highlighting}{Verbatim}{commandchars=\\\{\}}
% Add ',fontsize=\small' for more characters per line
\usepackage{framed}
\definecolor{shadecolor}{RGB}{248,248,248}
\newenvironment{Shaded}{\begin{snugshade}}{\end{snugshade}}
\newcommand{\KeywordTok}[1]{\textcolor[rgb]{0.13,0.29,0.53}{\textbf{#1}}}
\newcommand{\DataTypeTok}[1]{\textcolor[rgb]{0.13,0.29,0.53}{#1}}
\newcommand{\DecValTok}[1]{\textcolor[rgb]{0.00,0.00,0.81}{#1}}
\newcommand{\BaseNTok}[1]{\textcolor[rgb]{0.00,0.00,0.81}{#1}}
\newcommand{\FloatTok}[1]{\textcolor[rgb]{0.00,0.00,0.81}{#1}}
\newcommand{\ConstantTok}[1]{\textcolor[rgb]{0.00,0.00,0.00}{#1}}
\newcommand{\CharTok}[1]{\textcolor[rgb]{0.31,0.60,0.02}{#1}}
\newcommand{\SpecialCharTok}[1]{\textcolor[rgb]{0.00,0.00,0.00}{#1}}
\newcommand{\StringTok}[1]{\textcolor[rgb]{0.31,0.60,0.02}{#1}}
\newcommand{\VerbatimStringTok}[1]{\textcolor[rgb]{0.31,0.60,0.02}{#1}}
\newcommand{\SpecialStringTok}[1]{\textcolor[rgb]{0.31,0.60,0.02}{#1}}
\newcommand{\ImportTok}[1]{#1}
\newcommand{\CommentTok}[1]{\textcolor[rgb]{0.56,0.35,0.01}{\textit{#1}}}
\newcommand{\DocumentationTok}[1]{\textcolor[rgb]{0.56,0.35,0.01}{\textbf{\textit{#1}}}}
\newcommand{\AnnotationTok}[1]{\textcolor[rgb]{0.56,0.35,0.01}{\textbf{\textit{#1}}}}
\newcommand{\CommentVarTok}[1]{\textcolor[rgb]{0.56,0.35,0.01}{\textbf{\textit{#1}}}}
\newcommand{\OtherTok}[1]{\textcolor[rgb]{0.56,0.35,0.01}{#1}}
\newcommand{\FunctionTok}[1]{\textcolor[rgb]{0.00,0.00,0.00}{#1}}
\newcommand{\VariableTok}[1]{\textcolor[rgb]{0.00,0.00,0.00}{#1}}
\newcommand{\ControlFlowTok}[1]{\textcolor[rgb]{0.13,0.29,0.53}{\textbf{#1}}}
\newcommand{\OperatorTok}[1]{\textcolor[rgb]{0.81,0.36,0.00}{\textbf{#1}}}
\newcommand{\BuiltInTok}[1]{#1}
\newcommand{\ExtensionTok}[1]{#1}
\newcommand{\PreprocessorTok}[1]{\textcolor[rgb]{0.56,0.35,0.01}{\textit{#1}}}
\newcommand{\AttributeTok}[1]{\textcolor[rgb]{0.77,0.63,0.00}{#1}}
\newcommand{\RegionMarkerTok}[1]{#1}
\newcommand{\InformationTok}[1]{\textcolor[rgb]{0.56,0.35,0.01}{\textbf{\textit{#1}}}}
\newcommand{\WarningTok}[1]{\textcolor[rgb]{0.56,0.35,0.01}{\textbf{\textit{#1}}}}
\newcommand{\AlertTok}[1]{\textcolor[rgb]{0.94,0.16,0.16}{#1}}
\newcommand{\ErrorTok}[1]{\textcolor[rgb]{0.64,0.00,0.00}{\textbf{#1}}}
\newcommand{\NormalTok}[1]{#1}
\usepackage{graphicx,grffile}
\makeatletter
\def\maxwidth{\ifdim\Gin@nat@width>\linewidth\linewidth\else\Gin@nat@width\fi}
\def\maxheight{\ifdim\Gin@nat@height>\textheight\textheight\else\Gin@nat@height\fi}
\makeatother
% Scale images if necessary, so that they will not overflow the page
% margins by default, and it is still possible to overwrite the defaults
% using explicit options in \includegraphics[width, height, ...]{}
\setkeys{Gin}{width=\maxwidth,height=\maxheight,keepaspectratio}
\IfFileExists{parskip.sty}{%
\usepackage{parskip}
}{% else
\setlength{\parindent}{0pt}
\setlength{\parskip}{6pt plus 2pt minus 1pt}
}
\setlength{\emergencystretch}{3em}  % prevent overfull lines
\providecommand{\tightlist}{%
  \setlength{\itemsep}{0pt}\setlength{\parskip}{0pt}}
\setcounter{secnumdepth}{0}
% Redefines (sub)paragraphs to behave more like sections
\ifx\paragraph\undefined\else
\let\oldparagraph\paragraph
\renewcommand{\paragraph}[1]{\oldparagraph{#1}\mbox{}}
\fi
\ifx\subparagraph\undefined\else
\let\oldsubparagraph\subparagraph
\renewcommand{\subparagraph}[1]{\oldsubparagraph{#1}\mbox{}}
\fi

%%% Use protect on footnotes to avoid problems with footnotes in titles
\let\rmarkdownfootnote\footnote%
\def\footnote{\protect\rmarkdownfootnote}

%%% Change title format to be more compact
\usepackage{titling}

% Create subtitle command for use in maketitle
\newcommand{\subtitle}[1]{
  \posttitle{
    \begin{center}\large#1\end{center}
    }
}

\setlength{\droptitle}{-2em}
  \title{}
  \pretitle{\vspace{\droptitle}}
  \posttitle{}
  \author{}
  \preauthor{}\postauthor{}
  \date{}
  \predate{}\postdate{}


\begin{document}

\section{Reproduceacble Research}\label{reproduceacble-research}

\subsection{Library load}\label{library-load}

\begin{Shaded}
\begin{Highlighting}[]
\KeywordTok{library}\NormalTok{(ggplot2)}
\NormalTok{url <-}\StringTok{ "/Users/Bruker/Desktop/datascience/reproduceable_research/activity.csv"}
\NormalTok{activityData <-}\StringTok{ }\KeywordTok{read.csv}\NormalTok{(url)}
\KeywordTok{head}\NormalTok{(activityData)}
\end{Highlighting}
\end{Shaded}

\begin{verbatim}
##   steps       date interval
## 1    NA 2012-10-01        0
## 2    NA 2012-10-01        5
## 3    NA 2012-10-01       10
## 4    NA 2012-10-01       15
## 5    NA 2012-10-01       20
## 6    NA 2012-10-01       25
\end{verbatim}

\subsection{aggregate steps per day}\label{aggregate-steps-per-day}

\begin{Shaded}
\begin{Highlighting}[]
\NormalTok{steps <-}\StringTok{ }\KeywordTok{aggregate}\NormalTok{(activityData}\OperatorTok{$}\NormalTok{steps,}\DataTypeTok{by=}\KeywordTok{list}\NormalTok{(}\DataTypeTok{Date=}\NormalTok{activityData}\OperatorTok{$}\NormalTok{date),sum)}
\KeywordTok{head}\NormalTok{(steps)}
\end{Highlighting}
\end{Shaded}

\begin{verbatim}
##         Date     x
## 1 2012-10-01    NA
## 2 2012-10-02   126
## 3 2012-10-03 11352
## 4 2012-10-04 12116
## 5 2012-10-05 13294
## 6 2012-10-06 15420
\end{verbatim}

\subsection{adjust column names}\label{adjust-column-names}

\begin{Shaded}
\begin{Highlighting}[]
\KeywordTok{names}\NormalTok{(steps)[}\DecValTok{1}\NormalTok{]=}\StringTok{"Date"}
\KeywordTok{names}\NormalTok{(steps)[}\DecValTok{2}\NormalTok{]=}\StringTok{"totalSteps"}
\KeywordTok{head}\NormalTok{(steps)}
\end{Highlighting}
\end{Shaded}

\begin{verbatim}
##         Date totalSteps
## 1 2012-10-01         NA
## 2 2012-10-02        126
## 3 2012-10-03      11352
## 4 2012-10-04      12116
## 5 2012-10-05      13294
## 6 2012-10-06      15420
\end{verbatim}

\subsection{making histogram of total number of steps taken each
day}\label{making-histogram-of-total-number-of-steps-taken-each-day}

\begin{Shaded}
\begin{Highlighting}[]
\NormalTok{hist <-}\StringTok{ }\KeywordTok{ggplot}\NormalTok{(}\DataTypeTok{data=}\NormalTok{steps,}\KeywordTok{aes}\NormalTok{(}\DataTypeTok{x=}\NormalTok{totalSteps))}\OperatorTok{+}\KeywordTok{geom_histogram}\NormalTok{(}\DataTypeTok{binwidth=}\DecValTok{1000}\NormalTok{)}\OperatorTok{+}\KeywordTok{labs}\NormalTok{(}\DataTypeTok{title =} \StringTok{"Total  Daily Steps"}\NormalTok{, }\DataTypeTok{x =} \StringTok{"Steps"}\NormalTok{, }\DataTypeTok{y =} \StringTok{"Frequency"}\NormalTok{)}
\KeywordTok{print}\NormalTok{(hist)}
\end{Highlighting}
\end{Shaded}

\begin{verbatim}
## Warning: Removed 8 rows containing non-finite values (stat_bin).
\end{verbatim}

\includegraphics{week2_proj_files/figure-latex/unnamed-chunk-4-1.pdf}
\#\# Mean of steps taken per day

\begin{Shaded}
\begin{Highlighting}[]
\KeywordTok{mean}\NormalTok{(steps}\OperatorTok{$}\NormalTok{totalSteps,}\DataTypeTok{na.rm =} \OtherTok{TRUE}\NormalTok{)}
\end{Highlighting}
\end{Shaded}

\begin{verbatim}
## [1] 10766.19
\end{verbatim}

\subsection{median of steps taken per
day}\label{median-of-steps-taken-per-day}

\begin{Shaded}
\begin{Highlighting}[]
\KeywordTok{median}\NormalTok{(steps}\OperatorTok{$}\NormalTok{totalSteps,}\DataTypeTok{na.rm =} \OtherTok{TRUE}\NormalTok{)}
\end{Highlighting}
\end{Shaded}

\begin{verbatim}
## [1] 10765
\end{verbatim}

\subsection{what is average activity
patteren}\label{what-is-average-activity-patteren}

\subsection{1. Make a time series plot of the 5-minute interval (x-axis)
and the average number of steps taken, averaged across all days
(y-axis)}\label{make-a-time-series-plot-of-the-5-minute-interval-x-axis-and-the-average-number-of-steps-taken-averaged-across-all-days-y-axis}

\begin{Shaded}
\begin{Highlighting}[]
\NormalTok{five_min_int <-}\StringTok{ }\KeywordTok{aggregate}\NormalTok{(activityData}\OperatorTok{$}\NormalTok{steps,}\DataTypeTok{by=}\KeywordTok{list}\NormalTok{(activityData}\OperatorTok{$}\NormalTok{interval),mean,}\DataTypeTok{na.rm=}\OtherTok{TRUE}\NormalTok{)}
\KeywordTok{names}\NormalTok{(five_min_int)[}\DecValTok{1}\NormalTok{]=}\StringTok{"interval"}
\KeywordTok{names}\NormalTok{(five_min_int)[}\DecValTok{2}\NormalTok{]=}\StringTok{"steps"}
\KeywordTok{head}\NormalTok{(five_min_int)}
\end{Highlighting}
\end{Shaded}

\begin{verbatim}
##   interval     steps
## 1        0 1.7169811
## 2        5 0.3396226
## 3       10 0.1320755
## 4       15 0.1509434
## 5       20 0.0754717
## 6       25 2.0943396
\end{verbatim}

\begin{Shaded}
\begin{Highlighting}[]
\NormalTok{plot <-}\StringTok{ }\KeywordTok{ggplot}\NormalTok{(}\DataTypeTok{data =}\NormalTok{ five_min_int,}\KeywordTok{aes}\NormalTok{(}\DataTypeTok{x=}\NormalTok{interval,}\DataTypeTok{y=}\NormalTok{steps))}\OperatorTok{+}\KeywordTok{labs}\NormalTok{(}\DataTypeTok{title=} \StringTok{"Sum of Steps by Interval"}\NormalTok{, }\DataTypeTok{x =} \StringTok{"interval"}\NormalTok{, }\DataTypeTok{y =} \StringTok{"steps"}\NormalTok{)}\OperatorTok{+}\KeywordTok{geom_line}\NormalTok{(}\DataTypeTok{color=}\StringTok{"green"}\NormalTok{)}
\KeywordTok{print}\NormalTok{(plot)}
\end{Highlighting}
\end{Shaded}

\includegraphics{week2_proj_files/figure-latex/unnamed-chunk-7-1.pdf}
\#\#2. Which 5-minute interval, on average across all the days in the
dataset, contains the maximum number of steps

\begin{Shaded}
\begin{Highlighting}[]
\NormalTok{maxInterval <-}\StringTok{ }\NormalTok{five_min_int[}\KeywordTok{which.max}\NormalTok{(five_min_int}\OperatorTok{$}\NormalTok{steps),]}
\NormalTok{maxInterval}
\end{Highlighting}
\end{Shaded}

\begin{verbatim}
##     interval    steps
## 104      835 206.1698
\end{verbatim}

\subsection{Imputing missing values}\label{imputing-missing-values}

\subsection{1.Calculate and report the total number of missing values in
the
dataset}\label{calculate-and-report-the-total-number-of-missing-values-in-the-dataset}

\begin{Shaded}
\begin{Highlighting}[]
\NormalTok{missingSteps <-}\StringTok{ }\KeywordTok{is.na}\NormalTok{(activityData}\OperatorTok{$}\NormalTok{steps)}
\KeywordTok{head}\NormalTok{(missingSteps)}
\end{Highlighting}
\end{Shaded}

\begin{verbatim}
## [1] TRUE TRUE TRUE TRUE TRUE TRUE
\end{verbatim}

\begin{Shaded}
\begin{Highlighting}[]
\NormalTok{total_missing_values <-}\StringTok{ }\KeywordTok{sum}\NormalTok{(missingSteps)}
\NormalTok{total_missing_values}
\end{Highlighting}
\end{Shaded}

\begin{verbatim}
## [1] 2304
\end{verbatim}

\subsection{2.Devise a strategy for filling in all of the missing values
in the dataset. The strategy does not need to be sophisticated. For
example, you could use the mean/median for that day, or the mean for
that 5-minute interval,
etc.}\label{devise-a-strategy-for-filling-in-all-of-the-missing-values-in-the-dataset.-the-strategy-does-not-need-to-be-sophisticated.-for-example-you-could-use-the-meanmedian-for-that-day-or-the-mean-for-that-5-minute-interval-etc.}

\begin{Shaded}
\begin{Highlighting}[]
\KeywordTok{library}\NormalTok{(scales)}
\KeywordTok{library}\NormalTok{(Hmisc)}
\end{Highlighting}
\end{Shaded}

\begin{verbatim}
## Loading required package: lattice
\end{verbatim}

\begin{verbatim}
## Loading required package: survival
\end{verbatim}

\begin{verbatim}
## Loading required package: Formula
\end{verbatim}

\begin{verbatim}
## 
## Attaching package: 'Hmisc'
\end{verbatim}

\begin{verbatim}
## The following objects are masked from 'package:base':
## 
##     format.pval, units
\end{verbatim}

\begin{Shaded}
\begin{Highlighting}[]
\NormalTok{activityDataImputed <-}\StringTok{ }\NormalTok{activityData}
\NormalTok{activityDataImputed}\OperatorTok{$}\NormalTok{steps <-}\StringTok{ }\KeywordTok{impute}\NormalTok{(activityData}\OperatorTok{$}\NormalTok{steps,}\DataTypeTok{fun =}\NormalTok{ mean)}
\KeywordTok{head}\NormalTok{(activityDataImputed)}
\end{Highlighting}
\end{Shaded}

\begin{verbatim}
##     steps       date interval
## 1 37.3826 2012-10-01        0
## 2 37.3826 2012-10-01        5
## 3 37.3826 2012-10-01       10
## 4 37.3826 2012-10-01       15
## 5 37.3826 2012-10-01       20
## 6 37.3826 2012-10-01       25
\end{verbatim}

\subsection{3.Create a new dataset that is equal to the original dataset
but with the missing data filled
in.}\label{create-a-new-dataset-that-is-equal-to-the-original-dataset-but-with-the-missing-data-filled-in.}

\begin{Shaded}
\begin{Highlighting}[]
\NormalTok{stepsByDayImputed <-}\StringTok{ }\KeywordTok{aggregate}\NormalTok{(activityDataImputed}\OperatorTok{$}\NormalTok{steps,}\DataTypeTok{by=}\KeywordTok{list}\NormalTok{(activityDataImputed}\OperatorTok{$}\NormalTok{date),sum)}
\KeywordTok{names}\NormalTok{(stepsByDayImputed)[}\DecValTok{1}\NormalTok{]<-}\StringTok{ "Date"}
\KeywordTok{names}\NormalTok{(stepsByDayImputed)[}\DecValTok{2}\NormalTok{]<-}\StringTok{ "steps"}
\KeywordTok{head}\NormalTok{(stepsByDayImputed)}
\end{Highlighting}
\end{Shaded}

\begin{verbatim}
##         Date    steps
## 1 2012-10-01 10766.19
## 2 2012-10-02   126.00
## 3 2012-10-03 11352.00
## 4 2012-10-04 12116.00
## 5 2012-10-05 13294.00
## 6 2012-10-06 15420.00
\end{verbatim}

\subsection{4.Make a histogram of the total number of steps taken each
day and Calculate and report the mean and median total number of steps
taken per day. Do these values differ from the estimates from the first
part of the assignment? What is the impact of imputing missing data on
the estimates of the total daily number of
steps?}\label{make-a-histogram-of-the-total-number-of-steps-taken-each-day-and-calculate-and-report-the-mean-and-median-total-number-of-steps-taken-per-day.-do-these-values-differ-from-the-estimates-from-the-first-part-of-the-assignment-what-is-the-impact-of-imputing-missing-data-on-the-estimates-of-the-total-daily-number-of-steps}

\begin{Shaded}
\begin{Highlighting}[]
\NormalTok{hist <-}\StringTok{ }\KeywordTok{ggplot}\NormalTok{(}\DataTypeTok{data =}\NormalTok{ stepsByDayImputed,}\KeywordTok{aes}\NormalTok{(}\DataTypeTok{x=}\NormalTok{steps))}\OperatorTok{+}\KeywordTok{geom_histogram}\NormalTok{(}\DataTypeTok{binwidth =} \DecValTok{1000}\NormalTok{)}\OperatorTok{+}\KeywordTok{labs}\NormalTok{(}\DataTypeTok{title=}\StringTok{"Total Daily Steps (imputed)"}\NormalTok{,}\DataTypeTok{x=} \StringTok{"Total steps per day (Imputed)"}\NormalTok{,}\DataTypeTok{y=} \StringTok{"Frequency"}\NormalTok{)}
\KeywordTok{print}\NormalTok{(hist)}
\end{Highlighting}
\end{Shaded}

\includegraphics{week2_proj_files/figure-latex/unnamed-chunk-12-1.pdf}

\begin{Shaded}
\begin{Highlighting}[]
\KeywordTok{mean}\NormalTok{(stepsByDayImputed}\OperatorTok{$}\NormalTok{steps)}
\end{Highlighting}
\end{Shaded}

\begin{verbatim}
## [1] 10766.19
\end{verbatim}

\begin{Shaded}
\begin{Highlighting}[]
\KeywordTok{median}\NormalTok{(stepsByDayImputed}\OperatorTok{$}\NormalTok{steps)}
\end{Highlighting}
\end{Shaded}

\begin{verbatim}
## [1] 10766.19
\end{verbatim}

The mean appears to be unaffected by this simple data imputation. The
median is smaller.

\subsection{Are there differences in activity patterns between weekdays
and
weekends?}\label{are-there-differences-in-activity-patterns-between-weekdays-and-weekends}

\begin{Shaded}
\begin{Highlighting}[]
\NormalTok{activityDataImputed}\OperatorTok{$}\NormalTok{weekday <-}\StringTok{ }\KeywordTok{weekdays}\NormalTok{(}\KeywordTok{as.Date}\NormalTok{(activityDataImputed}\OperatorTok{$}\NormalTok{date))}
\NormalTok{activityDataImputed}\OperatorTok{$}\NormalTok{weekend<-}\StringTok{ }\KeywordTok{ifelse}\NormalTok{(activityDataImputed}\OperatorTok{$}\NormalTok{weekday}\OperatorTok{==}\StringTok{ "Saturday"} \OperatorTok{|}\StringTok{ }\NormalTok{activityDataImputed}\OperatorTok{$}\NormalTok{weekday}\OperatorTok{==}\StringTok{ "Sunday"}\NormalTok{ , }\StringTok{"weekend"}\NormalTok{, }\StringTok{"weekday"}\NormalTok{)}
\KeywordTok{head}\NormalTok{(activityDataImputed)}
\end{Highlighting}
\end{Shaded}

\begin{verbatim}
##     steps       date interval weekday weekend
## 1 37.3826 2012-10-01        0  Monday weekday
## 2 37.3826 2012-10-01        5  Monday weekday
## 3 37.3826 2012-10-01       10  Monday weekday
## 4 37.3826 2012-10-01       15  Monday weekday
## 5 37.3826 2012-10-01       20  Monday weekday
## 6 37.3826 2012-10-01       25  Monday weekday
\end{verbatim}

\begin{Shaded}
\begin{Highlighting}[]
\NormalTok{meanData <-}\StringTok{ }\KeywordTok{aggregate}\NormalTok{(activityDataImputed}\OperatorTok{$}\NormalTok{steps,}\DataTypeTok{by=}\KeywordTok{list}\NormalTok{(activityDataImputed}\OperatorTok{$}\NormalTok{weekend,activityDataImputed}\OperatorTok{$}\NormalTok{interval),mean)}
\KeywordTok{head}\NormalTok{(meanData)}
\end{Highlighting}
\end{Shaded}

\begin{verbatim}
##   Group.1 Group.2        x
## 1 weekday       0 7.006569
## 2 weekend       0 4.672825
## 3 weekday       5 5.384347
## 4 weekend       5 4.672825
## 5 weekday      10 5.139902
## 6 weekend      10 4.672825
\end{verbatim}

\begin{Shaded}
\begin{Highlighting}[]
\KeywordTok{names}\NormalTok{(meanData)[}\DecValTok{1}\NormalTok{]<-}\StringTok{ "weekend"}
\KeywordTok{names}\NormalTok{(meanData)[}\DecValTok{2}\NormalTok{]<-}\StringTok{ "interval"}
\KeywordTok{names}\NormalTok{(meanData)[}\DecValTok{3}\NormalTok{]<-}\StringTok{ "steps"}
\KeywordTok{ggplot}\NormalTok{(}\DataTypeTok{data =}\NormalTok{ meanData,}\KeywordTok{aes}\NormalTok{(}\DataTypeTok{x=}\NormalTok{interval,}\DataTypeTok{y=}\NormalTok{steps,}\DataTypeTok{color=}\NormalTok{weekend))}\OperatorTok{+}\KeywordTok{geom_line}\NormalTok{()}\OperatorTok{+}\KeywordTok{facet_grid}\NormalTok{(weekend }\OperatorTok{~}\StringTok{ }\NormalTok{.) }\OperatorTok{+}\StringTok{ }\KeywordTok{labs}\NormalTok{(}\DataTypeTok{title =} \StringTok{"Mean of Steps by Interval"}\NormalTok{, }\DataTypeTok{x =} \StringTok{"interval"}\NormalTok{, }\DataTypeTok{y =} \StringTok{"steps"}\NormalTok{)}
\end{Highlighting}
\end{Shaded}

\includegraphics{week2_proj_files/figure-latex/unnamed-chunk-13-1.pdf}
There seems to be variation in the beginning of the day during weekdays,
likely due to workplace activities. There seems to be an overall
slightly larger incidence of steps during the weekends.


\end{document}
